\documentclass[UTF8]{ctexart}
\usepackage[a4paper, top=25.4mm, bottom=25.4mm, left=31.8mm, right=31.8mm]{geometry}
\usepackage{graphicx}
\usepackage{amsmath}
\usepackage{multirow}
\usepackage{verbatim}
\usepackage{subcaption}
\usepackage{tabu}
\usepackage{booktabs}
\usepackage{minted}
\usemintedstyle{manni}
\usepackage[table]{xcolor}

\setlength{\parskip}{1em}
\definecolor{lightergray}{gray}{0.95}
\setlength{\tabcolsep}{12pt}
\renewcommand{\arraystretch}{1}

\begin{document}
\begin{titlepage}
  \begin{center}
    \vspace*{1cm}

    \Large
    编译原理

    \vspace{0.5cm}
    \Huge
    \textbf{语法分析实验实验报告}

    \vfill

    \normalsize\kaishu
    班级:07111603 \\
    学号:1120161730 \\
    姓名:武上博 \\
    \today
    \vspace{1cm}
  \end{center}
\end{titlepage}

\tableofcontents
\newpage

\section{实验目的}
\begin{enumerate}
  \item 熟悉 C 语言的语法规则,了解编译器语法分析器的主要功能
  \item 熟练掌握典型语法分析器构造的相关技术和方法,设计并实现具有一定分析能力的 C 语言语法分析器
  \item 掌握编译器从前端到后端各个模块的工作原理,语法分析模块与其他模块之间的交互过程
\end{enumerate}

\section{实验内容}
\begin{enumerate}
  \item 该实验选择 C 语言的一个子集,基于 BIT-MiniCC 构建 C 语法子集的语法分析器,该语法分析器能够读入 XML 文件形式的属性字符流,进行语法分析并进行错误处理,如果输入正确时输出 XML 形式的语法树,输入不正确时报告语法错误。
  \item 将分析树转换为抽象语法树,便于后续分析工作和代码生成工作的完成。
\end{enumerate}

\section{实验的具体过程步骤}

\section{运行效果}

\section{实验心得体会}

\end{document}